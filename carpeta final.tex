\documentclass[12pt,a4paper]{article}
\usepackage[utf8x]{inputenc}
\usepackage[spanish] {babel}
\usepackage{makeidx}
\usepackage[pdftex]{graphicx}


\title { \textbf{Trabajo Profesional}}
\date{2do cuatrimestre de 2010}
\author{\textbf{Ciancio Alessio, Mauro Lucas} \\
		\textbf{Gilioli, Leandro Ezequiel}	  \\
		\texttt{\{maurociancio,legilioli\}@gmail.com}
	}

\begin{document}
\maketitle
\tableofcontents
\newpage

	\section{Motivación}
Hoy en día el desarrollo de software en sus diversas tareas ha dejado de ser un trabajo 
puramente individual sino que requiere interacción de varias personas que en conjunto 
suman sus capacidades para lograr un producto de calidad superior.

Bajo esta forma de trabajo es necesario disponer de herramientas que ayuden a que los tiempos
requeridos por la coordinación e interacción de los integrantes de un grupo de trabajo sean bajos
de forma tal que el grupo sea productivo.

Con este objetivo, las herramientas utilizadas deben estar diseñadas e implementadas apuntando a
integrarse con las metodologías y herramientas existentes sin representar un obstáculo para el usuario
y a un bajo costo.

La idea nace de la necesidad de participar de una sesión de desarrollo, en la cual los participantes
no están físicamente en el mismo lugar. En algunos casos en particular, esta tarea requiere un
\textit{feedback} instantáneo entre los participantes que con herramientas ya existentes no se puede 
ofrecer.

Las herramientas existentes no logran que el feedback sea lo suficientemente rápido dado que trabajan
en un ambiente en el cuál es necesario respetar un protocolo para la modificación del estado de un 
documento. Por ejemplo: un usuario que está trabajando sobre un determinado archivo debe realizar todos
los cambios que quiere introducir, guardar la nueva versión del archivo modificado y enviarlo a sus
colaboradores para que estos esten al tanto de los cambios que introdujo (feedback). Mientras dura
este proceso los colaboradores no pueden realizar cambios al documento y si lo hacen será necesario
que realicen un proceso comunmente llamado \textit{merge}, de forma tal que el estado final del documento 
sea el mismo para cada uno de los participantes. Este proceso es lento y se hace especialmente engorroso
al aumentar el número de colaboradores.

Este esquema funciona bien en los casos en los cuales es baja la concurrencia sobre los mismos documentos,
es decir la edición de un mismo documento por parte de mas de un usuario es ocasional o en períodos de
tiempo disjuntos. Ejemplo: varios desarrolladores trabajando en un mismo proyecto con código fuente
compartido. Para estos casos existen herramientas cuya efectividad está comprobada, sistemas de control
de versiones (SVN, GIT, Mercurial) o servidores de archivos compartidos.

La solución desarrollada en este trabajo profesional resuelve los siguientes problemas:

\begin{itemize}
	\item Distribución geográfica: no es necesario estar en la misma ubicación física para que el proceso
	de desarrollo sea eficiente.
	\item Sin necesidad de un proceso de merge: el proceso de merge es realizado por el software en cada
	sitio de edición garantizando que el estado final del documento es el mismo para todos los 
	participantes. De esta manera se ahorra tiempo y se reducen los errores frecuentes o retrabajos
	los cuales son derivados de estos procesos.
	\item Alta latencia del feedback: los cambios en el estado del documento son reflejados en tiempo 
	real para todos los participantes.
\end{itemize}

La solución hace uso de tecnologías y herramientas existentes para proporcionar las funcionalidades 
que resuelven los problemas antes descriptos. Como ejemplo de esto, la solución se integró dentro de
Eclipse, el entorno de desarrollo integrado de facto para el lenguaje de programación Java.

	\section{Otras soluciones similares}
Durante la etapa de concepción del proyecto se analizaron otras soluciones similares al problema
anteriormente explicado. La descripción de cada una de ellas junto con la comparación de las mismas
respecto la presente solución fue detallada en el documento Propuesta de Trabajo Profesional. % cita aca %

A modo de resumen se presenta el cuadro comparativo \ref{soluciones_comparacion}:

\begin{table}[ht]
    \begin{tabular}{ | p{2.5cm} | p{5cm} | p{5cm} | }
    \hline
    Solución & Características & Comparativa \\ \hline

    Google Docs & Edicion de documentos en tiempo real desde un navegador. &
    Sólo puede utilizarse a través de un navegador e Internet. Código fuente cerrado. \\ \hline

    Google Wave & Comunicación y colaboración en tiempo real. &
    Idem Google Docs. El proyecto ha sido abandonado por Google. \\ \hline

    COLA (ECF) & Integración con Eclipse para colaboración en tiempo real de código fuente. &
	Limita a dos usuarios la cantidad de participantes en una sesión. Depende del proyecto ECF. \\ \hline

    BeWeeVee & Framework para integración de funcionalidades de colaboración en tiempo real para
    la plataforma .NET & Código fuente cerrado. Está desarrollado sólo para la plataforma .NET. \\ \hline

    \end{tabular}
    \caption{\label{soluciones_comparacion} Tabla comparativa de soluciones}
\end{table}

	\section{Seccion 2}

	\section{Seccion 3}

\newpage
\begin{thebibliography}{9}
	\bibitem{visiontpprof}
	Ciancio, Gilioli,
	\emph{Visión Trabajo Profesional}.
	Facultad de Ingeniería.
	Universidad de Buenos Aires. 

	\bibitem{jupiter}
	Nichols, Curtis, Dixon and Lamping,
	\emph{High-Latency, Low-Bandwidth Windowing in the Jupiter Collaboration System}.
	Xerox PARC.

	\bibitem{scrum}
	Ken Schwaber,
	\emph{Agile Software Development with Scrum}.
	Prentice Hall, 
	Octubre 2001.
	
	\bibitem{googledocs}
	\emph{Google Docs}. 
	Google Inc., 
	\textsl{http://docs.google.com}.
	
	\bibitem{googlewave}
	\emph{Google Wave}. 
	\textsl{http://wave.google.com}.

	\bibitem{beeweevee}
	Corvalius,
	\emph{BeWeeVee}. 
	\textsl{http://www.beweevee.com}.
	
	\bibitem{cola}
	Mustafa K. Isik,
	\emph{COLA}. 
	\textsl{ http://wiki.eclipse.org/RT\_Shared\_Editing }.
		
\end{thebibliography}

\end{document}